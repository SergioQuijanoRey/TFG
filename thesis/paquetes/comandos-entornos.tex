% DEFINICIÓN DE COMANDOS Y ENTORNOS

% CONJUNTOS DE NÚMEROS

\newcommand{\N}{\mathbb{N}}     % Naturales
\newcommand{\R}{\mathbb{R}}     % Reales
\newcommand{\Z}{\mathbb{Z}}     % Enteros
\newcommand{\Q}{\mathbb{Q}}     % Racionales
\newcommand{\C}{\mathbb{C}}     % Complejos

% TEOREMAS Y ENTORNOS ASOCIADOS

% \newtheorem{theorem}{Theorem}[chapter]
\newtheorem*{teorema*}{Teorema}
\newtheorem{teorema}{Teorema}[chapter]
\newtheorem{proposicion}{Proposición}[chapter]
\newtheorem{lema}{Lema}[chapter]
\newtheorem{corolario}{Corolario}[chapter]

\theoremstyle{definition}
\newtheorem{definicion}{Definición}[chapter]
\newtheorem{ejemplo}{Ejemplo}[chapter]

\theoremstyle{remark}
\newtheorem{observacion}{Observación}[chapter]

% MIS COMANDOS (Sergio Quijano)

% Comando que uso para expresar el conjunto {1, ..., n}
\newcommand{\deltaset}[1]{\Delta_{#1}}

% Para hacer referencias con el numero y el nombre de lo que estoy referenciando
\newcommand{\customref}[1]{''\ref{#1}, \nameref{#1}"}

% Espacio de matrices pxq
\newcommand{\matrices}[2]{\mathbb{M}_{#1 \times #2}}
