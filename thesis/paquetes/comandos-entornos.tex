% == Comandos de la plantilla ==
% ==============================================================================

\newcommand{\N}{\mathbb{N}}     % Naturales
\newcommand{\R}{\mathbb{R}}     % Reales
\newcommand{\Z}{\mathbb{Z}}     % Enteros
\newcommand{\Q}{\mathbb{Q}}     % Racionales
\newcommand{\C}{\mathbb{C}}     % Complejos

% TEOREMAS Y ENTORNOS ASOCIADOS

% \newtheorem{theorem}{Theorem}[chapter]:
\newtheorem*{teorema*}{Teorema}
\newtheorem{teorema}{Teorema}[chapter]
\newtheorem{proposicion}{Proposición}[chapter]
\newtheorem{lema}{Lema}[chapter]
\newtheorem{corolario}{Corolario}[chapter]

\theoremstyle{definition}
\newtheorem{definicion}{Definición}[chapter]
\newtheorem{ejemplo}{Ejemplo}[chapter]

\theoremstyle{remark}
\newtheorem{observacion}{Observación}[chapter]

% == Mis comandos propios ==
% ==============================================================================

% Comando que uso para expresar el conjunto {1, ..., n}
\newcommand{\deltaset}[1]{\Delta_{#1}}

% Para hacer referencias con el numero y el nombre de lo que estoy referenciando
\newcommand{\customref}[1]{''\ref{#1}, \nameref{#1}"}

% Espacio de matrices pxq
\newcommand{\espaciomatrices}[2]{\mathbb{M}_{#1 \times #2}}

% Espacio de tensores de orden N, #1 y de dimension M, #2 en cada modo
\newcommand{\espaciotensores}[2]{\mathcal{T}_{#1, #2}}

% Un mejor simbolo para QED
% Hago el renew para que \begin{proof} \end{proof} muestre este simbolo como a mi me gusta
\newcommand{\customqed}{\hfill\blacksquare}
\renewcommand{\qedsymbol}{$\customqed$}

% Elementos con explicaciones encima:

    % Igualdades
    \newcommand{\eqtext}[1]{\ensuremath{\stackrel{\text{#1}}{=}}}
    \newcommand{\eqmath}[1]{\ensuremath{\stackrel{#1}{=}}}

    % Implicaciones
    \newcommand{\impliestext}[1]{\ensuremath{\stackrel{\text{#1}}{\implies}}}

% Comando para indicar algun tipo de isomorfismo
\newcommand{\isomorfismo}[1]{\underset{#1}{\cong}}

% Expresiones logicas mas sencillas
\newcommand{\then}{\implies}
\newcommand{\iif}{\Longleftrightarrow}

% Algunos vectores que usamos bastante
\newcommand{\vectordd}[2]{\begin{pmatrix} #1 \\ #2 \end{pmatrix}}
\newcommand{\vectorn}[3]{\begin{pmatrix} #1 \\ #2 \\ \vdots \\ #3 \end{pmatrix}}

% Dos espacios
\newcommand{\dspace}{\ \ }

% Espacio de tensores de orden N (#1) y dimension M (#2) en cada modo
% TODO -- ahora estoy usando \espaciotensores que es mejor notacion
\DeclareMathOperator*{\MediumOtimes}{\text{\raisebox{0.25ex}{\scalebox{0.8}{$\bigotimes$}}}}
\newcommand{\esptensores}[2]{\overset{#1} \MediumOtimes \R^{#2}}

% Notacion para los vectores
% En el paper usan negrita, pero yo quiero usar una flechita encima
% Con este comand, si luego quiero cambiar la notación para los vectores (ie. volver
% a usar la notacion del paper) el cambio es sencillo
%
% NOTE: para poder usarse tiene que estar en un bloque de matematicas, por ejemplo:
% `$\nv{x}$`
\newcommand{\nv}[1]{\overrightarrow{#1}}

\newcommand{\entrecomillado}[1]{\textit{``#1''}}

% Forma mas comoda para escribir el producto escalar
\newcommand{\innerproduct}[2]{\langle #1 \; | \; #2 \rangle}

% Para añadir comentarios debajo o encima de elementos de una formula
\newcommand{\comentardebajo}[2]{\underbrace{#1}_\textrm{#2}}
\newcommand{\comentarencima}[2]{\overbrace{#1}^\textrm{#2}}
