
% Esto aparecia en '3.2.3 Eleccion de la famlilia de funciones de representacion'
% Pero J.Meri me indica: posponer, haz la elección del modelo y luego explica lo que pasa en la práctica con los parámetros del modelo
De hecho, en \cite{matematicas:principal} se prueba que para las Gaussianas y neuronas es suficiente con tomar un número finito de funciones ${f_1, \ldots, f_M}$. Es más, en el caso de estar trabajando con imágenes, proponen que basta con tomar $M$ entorno a 100.

% Esto en la introducción del capítulo 4
% J.Meri me indica: primero haz los modelos y luego habla de sus propiedades
Además, el modelo profundo es equivalente a las redes conocidas como \textit{SimNets} \cite{matematicas:principal}, lo que reafirma el hecho de que la modelización es muy buena.
