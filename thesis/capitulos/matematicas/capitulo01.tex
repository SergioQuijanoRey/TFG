% !TeX root = ../libro.tex
% !TeX encoding = utf8

\chapter{Introducción}\label{ch:primer-capitulo}

El objetivo de este trabajo, desde la perspectiva de las matemáticas, es analizar las redes neuronales profundas, basadas en convoluciones, y explicar por qué estas funcionan mejor que las redes neuronales no profundas, conocidas en la literatura como redes neuronales \textit{``shallow"}.

% TODO -- esta frase es muy mala
El desarrollo que haremos en esta parte se puede resumir tal que:

\begin{enumerate}
    \item Definir un espacio de hipótesis en el que es fundamental el uso de tensores
    \item Implementar dicho espacio de hipótesis con redes profundas y no profundas, usando en cada caso una descomposición tensorial concreta (descomposición \textit{CP} para redes no profundas, descomposición \textit{HT} para redes profundas)
    \item Demostrar dos resultados centrales que nos muestran la superioridad de las redes profundas, en lo que se conoce como \textit{depth efficiency}
\end{enumerate}

Nos basaremos principalmente en el trabajo \cite{matematicas:principal}


\endinput
