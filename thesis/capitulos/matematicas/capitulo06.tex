\chapter{Conclusiones y trabajo futuro} \label{chapter:conclusiones_trabajo_futuro}

En este trabajo hemos conseguido modelar matemáticamente la tarea de aprendizaje que se realiza en la práctica del aprendizaje automático, describir dos modelos matemáticos que se asemejan muchísimo a las arquitecturas usadas en la práctica del aprendizaje automático. Con estas modelizaciones, hemos conseguido estudiar la mayor capacidad expresiva de las redes profundas frente a las redes no profundas, y dar información precisa sobre cómo de frecuente ocurre este hecho. Por lo tanto, \textbf{consideramos que los objetivos iniciales del trabajo se han cumplido de forma satisfactoria}.

Sin embargo, nos hemos centrado en estudiar redes convolucionales profundas y no profundas, que se han usado sobre todo en tareas de visión por computador. No hemos estudiado otras arquitecturas más actuales usadas para visión, como pueden ser las \textit{redes generativas adversarias} o \textit{GANs}. O arquitecturas basadas en mecanismos de atención, como los \textit{Transformers}, que están teniendo mucho protagonismo tanto en tareas de visión como en otros ámbitos (principalmente en procesamiento de lenguaje natural o \textit{NLP}).

Y aunque hayamos realizado una modelización muy buena de las redes convolucionales profundas y no profundas, existen algunos elementos técnicos muy presentes en estas redes que no hemos considerado. Por ejemplo, en el modelo \textit{ResNet} (que hemos usado en la parte informática del presente trabajo), se utiliza el \todo{completar}

Por lo tanto, algunas mejoras y futuros trabajos sobre los que expandir el presente trabajo son:

\begin{itemize}
	\item Introducir elementos técnicos como capas residuales, capas de salto, normalización de \textit{batches} en nuestras modelizaciones. Estudiar cómo impactan estos elementos en la capacidad expresiva de las redes.
	\item Estudiar una posible modelización usando descomposiciones tensoriales para las arquitecturas que hemos mencionado previamente.
	\item Probar ciertos resultados en base a las modelizaciones que nos den información sobre su mayor potencia expresiva respecto a otros modelos.
\end{itemize}
