\chapter{Herramientas matemáticas fundamentales}

\section{Tensores}

\subsection{Notación}

Seguiremos en parte la notación del trabajo principal \cite{matematicas:principal}, aunque introducimos pequeños cambios.

A los vectores con tipografía en negrita, tal que $\mathbf{v} \in \R^N$. Las coordenadas de dicho vector se denotarán como $v_i$ con $i \in \deltaset{n}$, donde $\deltaset{n} := \{1, \ldots, n\}$.

Para denotar a los tensores (que introduciremos más adelante), usaremos la tipografía caligráfica. Por ejemplo $\mathcal{A} \in \R^{M_1 \times \ldots M_N}$

\newpage
\section{Teoría de la medida}

\endinput
