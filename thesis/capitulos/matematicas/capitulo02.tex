\chapter{Herramientas matemáticas fundamentales}

\section{Tensores}

\subsection{Notación}

Seguiremos en parte la notación del trabajo principal \cite{matematicas:principal}, aunque introducimos pequeños cambios.

A los vectores con tipografía en negrita, tal que $\mathbf{v} \in \R^N$. Las coordenadas de dicho vector se denotarán como $v_i$ con $i \in \deltaset{n}$, donde $\deltaset{n} := \{1, \ldots, n\}$.

Para denotar a los tensores (que introduciremos más adelante), usaremos la tipografía caligráfica. Por ejemplo $\mathcal{A} \in \R^{M_1 \times \ldots M_N}$. Cada una de las entradas de dicho tensor (más adelante, en \customref{sec:deftensor}, veremos que significa esto) serán denotadas como $\mathcal{A}_{d_1, \ldots, d_N} \in \R$.

Al espacio de matrices de dimensiones $p, q$ lo denotaremos, como es usual, como $\matrices{p}{q}$.

\subsection{Definición del producto tensorial} \label{sec:deftensor}

\subsection{Otra forma de ver los tensores}

% TODO -- especificar que es lo que hemos visto anteriormente
Por lo visto anteriormente, tenemos una forma más concreta de entender los tensores.

Podemos ver un tensor $\mathcal{A} \in \R^{M_1, \ldots, M_N}$ como un \textit{array} multidimensional. La siguiente notación será muy útil a lo largo de este trabajo:

\begin{itemize}
    \item Modos: cada una de las entradas $d_1, \ldots, d_N$ que podemos usar para indexar los elementos del tensor
    \item Orden: el número de modos del tensor. En el caso de nuestro tensor $\mathcal{A}$, tenemos $N$ modos, y por tanto ese es su orden
    \item Dimensión: el número de valores que puede tomar cada uno de los modos. Por lo tanto, si en el primer modo $d_i$ puede tomar valores en $\deltaset{M}$, diremos que el modo $i$-ésimo tiene dimensión $M$.
        \begin{itemize}
            \item Un tensor puede tener distintas dimensiones en cada uno de los modos, o tener la misma dimensión para todos los modos
            \item Por tanto, sería más correcto hablar de \textit{"dimensiones de los modos"} que de \textit{"dimensión de un tensor"}, pero en ocasiones se abusa del lenguaje
        \end{itemize}
\end{itemize}

Ahora, respecto al \textbf{producto tensorial}, por los isomorfismos que hemos introducido previamente (TODO -- hay que desarrollar estos isomorfismos), podemos ver el producto tensorial entre dos tensores reales de una forma más sencilla.

Sean $\mathcal{A}, \mathcal{B}$ dos tensores de órdenes $P, Q$ respectivamente. Entonces el producto tensorial de estos dos, que ya sabemos que se denota como $\mathcal{A} \otimes \mathcal{B}$, es un tensor de orden $P + Q$ cuyos elementos se pueden expresar como:

$$(A \otimes B)d_1, \ldots d_{P + Q} = A_{d_1, \ldots, d_P} \cdot B_{d_{P + 1}, \ldots, d_{P + Q}}$$

Por tanto, es claro que en el caso de que tengamos dos tensores de orden 1, $u \in \R^{M_1}, v \in \R^{M_2}$, su producto tensorial se corresponde con la operación $u v^T \in \matrices{M_1}{M_2}$ donde $v^T$ indica la traspuesta de $u$.

\newpage
\section{Teoría de la medida}

\endinput
