% !TeX root = ../../libro.tex
% !TeX encoding = utf8

\chapter{Herramientas matemáticas fundamentales} \label{ch:matematicas_fundamentales}

\section{Tensores}

\subsection{Notación}

Seguiremos en parte la notación del trabajo principal \cite{matematicas:principal}, aunque introducimos pequeños cambios.

Denotaremos a los vectores con tipografía en negrita, tal que $\mathbf{v} \in \R^N$. Las coordenadas de dicho vector se denotarán como $v_i$ con $i \in \deltaset{n}$, donde $\deltaset{n} := \{1, \ldots, n\}$.

Para denotar a los tensores (que introduciremos más adelante), usaremos la tipografía caligráfica. Por ejemplo $\mathcal{A} \in \R^{M_1 \times \ldots M_N}$. Cada una de las entradas de dicho tensor (más adelante, en \customref{sec:deftensor}, veremos que significa esto) serán denotadas como $\mathcal{A}_{d_1, \ldots, d_N} \in \R$.

Al espacio de matrices de dimensiones $p, q$ lo denotaremos, como es usual, como $\matrices{p}{q}$.

Al producto tensorial entre dos tensores $\mathcal{A}, \mathcal{B}$ lo denotaremos como $\mathcal{A} \otimes \mathcal{B}$. Dado un conjunto de vectores $\mathbf{v}_1, \ldots, \mathbf{v}_N$, denotaremos su producto tensorial $\mathbf{v}_1 \otimes \ldots \otimes {v}_N$ como $\otimes_{i = 1}^N \mathbf{v}_i$.

Al producto de Kronecker entre dos matrices $A, B$ lo denotaremos como $A \odot B$.

% TODO -- tengo que referenciar o bien los videos que he seguido o un texto que haga un desarollo similar
\subsection{Definición del producto tensorial} \label{sec:deftensor}

Dados dos espacios vectoriales reales (aunque podría realizarse la construcción sobre otro cuerpo) $\mathbb{V}, \mathbb{W}$, queremos construir el espacio producto tensorial de estos espacios vectoriales, denotado como $\mathbb{V} \otimes \mathbb{W}$. Buscamos que este nuevo objeto matemático tenga propiedades similares a las del producto entre escalares, principalmente la propiedad distributiva y la propiedad asociativa. Especificaremos esto en \customref{sec:cociente_prod_formal}

Para la construcción del producto tensorial de espacios vectoriales, necesitaremos primero introducir conceptos previos que serán fundamentales en la construcción del objeto matemático que buscamos.

\subsubsection{Producto formal de dos espacios vectoriales}

% TODO -- la notacion span es americana, buscar la española en los apuntes de
% geometria I
\begin{definicion}[Producto formal de dos espacios vectoriales]
    Sean $\mathbb{V}, \mathbb{W}$ dos espacios vectoriales reales. Se define su \textbf{producto formal} como:

    \begin{equation}
        \mathbb{V} \ast \mathbb{W} := span_{\mathbb{R}} \{v \ast w / v \in \mathbb{V}, w \in \mathbb{W} \}
    \end{equation}

    Donde $span$ denota el conjunto formado por todas las combinaciones lineales finitas de los elementos del conjunto, es decir,

    \begin{equation}
        span_{\mathbb{R}}(A) := \{ \sum_{k = 1}^n \alpha_i a_i / n \in \N, \alpha_i \in \R, a_i \in A \}
    \end{equation}

    Hay que tener en cuenta que en $v \ast w$, $\ast$ es un símbolo con el que no sabemos operar. Por tanto, ahora mismo no sabemos simplificar muchas expresiones en este espacio.
\end{definicion}

Es claro que por ser $\mathbb{V}, \mathbb{W}$ espacios vectoriales, y estar tomando combinaciones lineales finitas, $\mathbb{V} \ast \mathbb{W}$ es un espacio vectorial.

Veamos ahora algunos ejemplos para familiarizarnos con este espacio.

\begin{ejemplo} \label{ej:prod_formal}

Sea $\Omega := R^2 \ast R^3$. Entonces tenemos que:

\begin{equation}
    \Omega = span \{
        \begin{pmatrix}
            a \\
            b
        \end{pmatrix}
        \ast
        \begin{pmatrix}
            x \\
            y \\
            z
        \end{pmatrix}
        / a, b, x, y, z \in \R
    \}
\end{equation}

Un elemento de dicho espacio puede ser

$$
2 \left[ \begin{pmatrix}1 \\ 2\end{pmatrix} \ast \begin{pmatrix}3 \\ 2 \\ 1 \end{pmatrix} \right]
- 3 \left[ \begin{pmatrix}1 \\ 0\end{pmatrix} \ast \begin{pmatrix}1 \\ 2 \\ 3 \end{pmatrix} \right]
$$

Notar que no sabemos cómo meter los escalares $2, -3$ en los elementos del producto formal.

Por cómo hemos definido el producto formal de dos espacios vectoriales, tenemos que $dim(\R^2 \ast \R^3) = \infty$

\end{ejemplo}

\subsubsection{Producto tensorial a partir del producto formal} \label{sec:cociente_prod_formal}

Para motivar el nuevo objeto que vamos a construir, hay que tener en cuenta que ciertas propiedades deseables no se cumplen, en general, en el producto formal:

\begin{enumerate}
    \item $c [v \ast w] = (cv) \ast w$
    \item $c[v \ast w] = v \ast (cw)$
    \item $(v_1 + v_2) \ast w = v_1 \ast w + v_2 \ast w$
    \item $v \ast (w_1 + w_2) = v \ast w_1 + v \ast w_2$

    Donde estamos tomando $v, v_1, v_2 \in \mathbb{V}, w, w_1, w_2 \in \mathbb{W}$
\end{enumerate}

Estas igualdades representan las propiedades que queremos que se cumplan, para que nuestro nuevo objeto matemático tenga un comportamiento similar al del producto entre escalares. En concreto, 1 y 2 nos dan una especie de asociatividad, mientras que 3 y 4 nos da la propiedad distributiva.

Como $\mathbb{V} \ast \mathbb{W}$ es un espacio vectorial, podemos usar el espacio cociente para introducir estas propiedades. Para ello definimos:

\begin{equation}
\begin{split}
    I := span\{& \\
               & (cv) \ast w - c(v \ast w), \\
               & v \ast (cw) - c(v \ast w), \\
               & (v_1 + v_2) \ast w - (v_1 \ast w + v_2 \ast w), \\
               & v \ast (w_1 + w_2) - (v \ast w_1 + v \ast w_2), \\
\}&
\end{split}
\end{equation}

Con esto, ya podemos definir el producto tensorial:

\begin{definicion}[Producto tensorial]
    Dados dos espacios vectoriales $\mathbb{V}, \mathbb{W}$, se define su \textbf{producto tensorial} como:

    $$\mathbb{V} \otimes \mathbb{W} := (\mathbb{V} \ast \mathbb{W}) / I$$

    con lo que dados $v \in \mathbb{V}, w \in \mathbb{W}$, tenemos que

    $$v \otimes w = v \ast w + I$$
\end{definicion}

A partir de esta definición, son directas las siguientes propiedades:

% TODO -- creo que falta la distributividad respecto (v1 * v2) * w
\begin{proposicion} \label{prop:tensores_propiedades}
    Sean $v, v_1, v_2 \in \mathbb{V}, w \in \mathbb{W}, \lambda \in \R$, entonces son ciertas:
    \begin{enumerate}
        \item $\lambda [v \otimes w] = (\lambda v) \otimes w$
        \item $\lambda [v \otimes w] = v \otimes (\lambda w)$
        \item $v \otimes (w_1 + w_2) = v \otimes w_1 + v \otimes w_2$
    \end{enumerate}
\end{proposicion}

\begin{proof} $\newline \newline$
    1.

    \begin{equation}
    \begin{split}
        (cv) \otimes w &\eqtext{def} (cv) \ast w + I = \ldots \ \text{usando que} \quad a + I = a + i + I, \quad \forall i \in I \\
        \ldots &= (cv) \ast w + (c(v \ast w) - cv \ast w) + I = \ldots \\
        \ldots &= \cancel{(cv) \ast w} + c(v \ast w) - \cancel{cv \ast w} + I = \ldots \\
        \ldots &= c(v \ast w) + I = c (v \otimes w) \\
    \end{split}
    \end{equation}

    $\customqed$
    $\newline$
    2. TODO
    $\newline$
    3. TODO
\end{proof}

También sabemos que por ser $\mathbb{V} \ast \mathbb{W}$ un espacio vectorial, y al definir el producto tensorial como un cociente, $\mathbb{V} \otimes \mathbb{W}$ es también un espacio vectorial.

Ahora, enunciamos un importante teorema que nos ayudará a entender la naturaleza del producto vectorial:

\begin{teorema}[Base del espacio vectorial \textit{producto tensorial}] \label{th:base_prod_tensorial}
    Sean $\mathbb{B}_{\mathbb{V}} = \{v_1, \ldots, v_n\}$, $\mathbb{B}_{\mathbb{W}} = \{w_1, \ldots, w_m\}$ bases de $\mathbb{V}, \mathbb{W}$ respectivamente, entonces:

    $$\mathbb{B}_{\mathbb{V} \otimes \mathbb{W}} := \{v_i \otimes w_j / i \in \deltaset{n}, j \in \deltaset{m}\}$$

    es una base del espacio vectorial $\mathbb{V} \otimes \mathbb{W}$, y por lo tanto:

    $$dim(\mathbb{V} \otimes \mathbb{W}) = dim(\mathbb{V}) \cdot dim(\mathbb{W})$$
\end{teorema}

\begin{proof} $\newline$
    % En el video esta la prueba, pero no la tengo copiada
    TODO -- escribir esta prueba
    TODO -- en la página 20 de mis notas tengo apuntada la prueba de que es un sistema de generadores. Falta ver la independencia

\end{proof}

Notar que podríamos haber usado este teorema como forma de definir el producto tensorial de dos espacios vectoriales. Sin embargo, limitaríamos esta construcción a espacios vectoriales que admitiesen una base.

Una consecuencia inmediata es que todo tensor $\gamma \in \mathbb{V} \otimes \mathbb{W}$ se escribe de la forma:

\begin{equation}
    \gamma = \sum_{\substack{v_i \in \mathbb{V}\\ w_i \in \mathbb{W}\\ i \in \deltaset{n}}} c_{v_i, w_i} \cdot v_i \otimes w_i, \qquad c_{v_i, w_i} \in \R
\end{equation}

La expresión anterior motiva la siguiente definición:

\begin{definicion}[Tensor puro]
    Un tensor $\gamma \in \mathbb{V} \otimes \mathbb{W}$ se dice puro cuando existen $v \in \mathbb{V}, w \in \mathbb{W}$ tales que $\gamma = v \otimes w$.
\end{definicion}

Otra consecuencia del \customref{th:base_prod_tensorial} es que, en general, $v \otimes w \neq w \otimes v$, en el caso en el que tengan sentido las dos operaciones.

Veamos ahora otra propiedad interesante. Queremos que el producto tensorial se asemeje al producto entre escalares. Para ello, sería natural que $v \otimes \vv{0_w} = \vv{0_{\mathbb{V} \otimes \mathbb{W}}}$.

\begin{proposicion}
    Sean $\mathbb{V}, \mathbb{W}$ espacios vectoriales sobre $\R$. Sean $v \in \mathbb{V}, w \in \mathbb{W}$. Entonces se verifica:

    \begin{enumerate}
        \item $v \otimes \vv{0_\mathbb{W}} = \vv{0_{\mathbb{V} \otimes \mathbb{W}}}$
        \item $\vv{0_{\mathbb{V}}} \otimes w = \vv{0_{\mathbb{V} \otimes \mathbb{W}}}$
    \end{enumerate}
\end{proposicion}
\begin{proof}
    Empezamos con la primera igualdad. Sabemos que en un espacio vectorial (como es el caso del producto tensorial) se verifica que:

    \begin{equation} \label{eq:dem_tensor_cero}
        \vv{v} + \vv{w} = \vv{w} \then \vv{v} = \vv{0}
    \end{equation}

    Veamos esto ahora con nuestro primer candidato a cero del espacio vectorial producto tensorial:

    \begin{equation}
    \begin{split}
        v \otimes \vv{0_\mathbb{W}} + v \otimes w \eqtext{3.} v \otimes (\vv{0_\mathbb{W}} + w) = v \otimes (w)
    \end{split}
    \end{equation}

    La demostración para $\vv{0_{\mathbb{V}}} \otimes w = \vv{0_{\mathbb{V} \otimes \mathbb{W}}}$ es completamente análoga.

    % TODO -- en la pagina 18 de mis apuntes veo otra forma de demostrar esto, usando las propiedades 1, 2 en vez de usando la 3

\end{proof}

Con esto podemos pasar a ver algunos ejemplos:

\begin{ejemplo}
    Sea $\Omega := \R^2 \otimes \R^3$.

    En primer lugar, sabemos que $dim(\Omega) = 2 \cdot 6$. Por ser un espacio vectorial real, de dimensión $6$, sabemos que $\Omega \cong \R^6$.

    Ahora, consideramos las bases usuales de $\R^2, \R^3$, y usando el teorema \ref{th:base_prod_tensorial}, construimos una base, que podríamos considerar la base usual para el producto tensorial:

    \begin{equation}
    \begin{split}
    \mathbb{B}_{\Omega} = \{& \\
        & \begin{pmatrix}1 \\ 0 \end{pmatrix} \otimes \begin{pmatrix} 1 \\ 0 \\ 0 \end{pmatrix},
        \begin{pmatrix}1 \\ 0 \end{pmatrix} \otimes \begin{pmatrix} 0 \\ 1 \\ 0 \end{pmatrix},
        \begin{pmatrix}1 \\ 0 \end{pmatrix} \otimes \begin{pmatrix} 0 \\ 0 \\ 1 \end{pmatrix}, \\
        & \begin{pmatrix}0 \\ 1 \end{pmatrix} \otimes \begin{pmatrix} 1 \\ 0 \\ 0 \end{pmatrix},
        \begin{pmatrix}0 \\ 1 \end{pmatrix} \otimes \begin{pmatrix} 0 \\ 1 \\ 0 \end{pmatrix},
        \begin{pmatrix}0 \\ 1 \end{pmatrix} \otimes \begin{pmatrix} 0 \\ 0 \\ 1 \end{pmatrix} \\
    & \}
    \end{split}
    \end{equation}

    Ahora, retomamos una operación del \ref{ej:prod_formal} (solo que esta vez con el producto tensorial, y no con el producto formal):

    $$
    2 \left[ \begin{pmatrix}1 \\ 2\end{pmatrix} \otimes \begin{pmatrix}3 \\ 2 \\ 1 \end{pmatrix} \right]
    - 3 \left[ \begin{pmatrix}1 \\ 0\end{pmatrix} \otimes \begin{pmatrix}1 \\ 2 \\ 3 \end{pmatrix} \right]
    $$

    % TODO -- desarrollar esta expresion
    Tenemos distintas formas para desarrollar esta expresión, por ejemplo:

    TODO -- desarrollar algo esta expresión

\end{ejemplo}

A partir de lo anterior, podemos ir abstrayendo ciertas propiedades resultantes de hacer el producto tensorial, siendo uno de los dos espacios un cierto $\R^N$. Empezamos con el siguiente ejemplo:

\begin{ejemplo}
    Sea $\mathbb{V}$ un espacio vectorial real y considero $\Omega := \R \otimes \mathbb{V}$.

    Comienzo considerando un tensor de la forma $\gamma = a(x \otimes u) + b(y \otimes v)$ con $a, b \in \R, x, y \in \R, u, v \in \mathbb{V}$. Usando las propiedades de los tensores (\ref{prop:tensores_propiedades}), desarrollo esta expresión:

    \begin{equation}
    \begin{split}
        a(x \otimes u) + b(y \otimes v) &\eqtext{2.} \\
        = & x \otimes (au) + y \otimes (bv) \eqtext{2} \ldots \text{ usando que }  x,y \in \R \ldots \\
        \ldots = & 1 \otimes ((ax) u) + 1 \otimes ((by) v) \eqtext{3.} \\
        = & 1 \otimes ((ax)u + (by) v) = \\
        = & 1 \otimes w \quad \text{con } w := (ax)u + (by), \quad \text{luego }w \in \mathbb{V}
    \end{split}
    \end{equation}

    Esto sirve como prueba del siguiente resultado:

    \begin{proposicion}
        Sea $\gamma \in \R \otimes \mathbb{V}$ con $\mathbb{V}$ un espacio vectorial real. Entonces:

        \begin{equation}
            \gamma \text{ tensor puro} \implies \exists v \in \mathbb{V}: \gamma = 1 \otimes v
        \end{equation}
    \end{proposicion}

    Buscamos extender este resultado para un tensor cualquiera (no necesariamente puro) del espacio $\R \otimes \mathbb{V}$:

    \begin{proposicion}
        Sea $\gamma \in \R \otimes \mathbb{V}$. Entonces $\exists v \in \mathbb{V}: \gamma = 1 \otimes v$
    \end{proposicion}
    \begin{proof}
        Usando \ref{th:base_prod_tensorial}, podemos expresar:

        \begin{equation}
            \gamma = \sum_{\substack{v_i \in \mathbb{V}\\ w_i \in \mathbb{W}\\ i \in \deltaset{n}}} c_{v_i, w_i} \cdot v_i \otimes w_i, \qquad c_{v_i, w_i} \in \R
        \end{equation}

        Usando ahora la proposición anterior en los tensores puros de la sumatoria, queda:

        \begin{equation}
        \begin{split}
            \gamma &= \sum_{\substack{w_i \in \mathbb{W}\\ i \in \deltaset{n}}} c_{w_i} \cdot 1 \otimes w_i \eqtext{2.} \\
            & = \sum_{\substack{w_i \in \mathbb{W}\\ i \in \deltaset{n}}} 1 \otimes (c_{w_i} \cdot w_i) \eqtext{3.} \\
            & = 1 \otimes ( \sum_{\substack{w_i \in \mathbb{W} \\ i \in \deltaset{n}}} c_{w_i} \cdot v_i ) = \\
            & = 1 \otimes w \qquad \text{con } w := \sum_{\substack{w_i \in \mathbb{W}\\ i \in \deltaset{n}}} c_{w_i} w_i \implies w \in \mathbb{V}
        \end{split}
        \end{equation}

    \end{proof}


    Con esto, podemos pasar al siguiente resultado que nos permitirá ver de una forma mucho más clara en que espacio nos encontramos:

    \begin{proposicion}
        $\R \otimes \mathbb{V} \isomorfismo{\text{vec}} \mathbb{V}$
    \end{proposicion}
    \begin{proof}
        Basta con considerar

        \begin{equation}
        \begin{split}
            \phi: \R \otimes \mathbb{V} &\to \mathbb{V} \\
            v = 1 \otimes w \in \mathbb{V} &\mapsto w
        \end{split}
        \end{equation}

        Veamos, aunque sea prácticamente inmediato, que $\phi$ es biyectiva y lineal:

        Inyectividad: $\phi(1 \otimes w_1) = \phi(1 \otimes w_2) \underset{def.\ \phi}{\iif} w_1 = w_2$

        Sobreyectividad: $w \in \mathbb{V} \then \phi(1 \otimes w) = w$

        Linealidad 1. $\phi(1 \otimes w_1 + 1 \otimes w_2) \eqtext{3.} \phi(1 \otimes (w_1 + w_2)) = w_1 + w_2$

        Linealidad 2. $\phi(\lambda (1 \otimes w)) \eqtext{2.} \phi(1 \otimes (\lambda w)) = \lambda w$

    \end{proof}
\end{ejemplo}

\begin{ejemplo} \label{ejemplo:R2xR2}
    Consideramos ahora el espacio $\Omega := \R^2 \otimes R^2$. Por tanto, sabemos que los tensores puros de este espacio son de la forma:

    $$\begin{pmatrix} a \\ b \end{pmatrix} \otimes \begin{pmatrix} c \\ d \end{pmatrix}$$

    Tomando ahora la base usual de $\R^2$, cualquier tensor de $\gamma \in \Omega$ se puede expresar como:

    \begin{equation}
    \begin{split}
        \gamma &= \lambda_{11} \vectordd{1}{0} \otimes \vectordd{1}{0} + \lambda_{12} \vectordd{1}{0} \otimes \vectordd{0}{1} + \ldots \\
        \ldots &+ \lambda_{21} \vectordd{0}{1} \otimes \vectordd{1}{0} + \lambda_{22} \vectordd{0}{1} \otimes \vectordd{0}{1}
    \end{split}
    \end{equation}

    De nuevo, sabemos que $\R^2 \otimes \R^2 \cong \R^4$. La expresión anterior, sin embargo, nos invita a considerar el espacio $\Omega$ como $\R^2 \otimes \R^2 \cong \R^4 \cong \matrices{2}{2}$, con lo que con un adecuado isomorfismo, podemos expresar:

    $$\gamma = \begin{bmatrix}
        \lambda_{11} & \lambda_{12} \\
        \lambda_{21} & \lambda_{22}
    \end{bmatrix}$$

    Para ello basta con considerar el isomorfismo:

    \begin{equation}
        \phi: \R \otimes \mathbb{V} \to \mathbb{V}
    \end{equation}

    de forma que:

    $$\phi(\vectordd{1}{0} \otimes \vectordd{1}{0}) = \begin{bmatrix} 1 & 0 \\ 0 & 0 \end{bmatrix}$$
    $$\phi(\vectordd{1}{0} \otimes \vectordd{0}{1}) = \begin{bmatrix} 0 & 1 \\ 0 & 0 \end{bmatrix}$$
    $$\phi(\vectordd{0}{1} \otimes \vectordd{1}{0}) = \begin{bmatrix} 0 & 0 \\ 1 & 0 \end{bmatrix}$$
    $$\phi(\vectordd{0}{1} \otimes \vectordd{0}{1}) = \begin{bmatrix} 0 & 0 \\ 0 & 1 \end{bmatrix}$$

    A partir de este isomorfismo, podemos dar cierto significado al producto de tensores puros:

    \begin{equation}
    \begin{split}
        \gamma &:= \vectordd{a}{b} \otimes \vectordd{c}{d} = \left( a \cdot \vectordd{1}{0} + b \cdot \vectordd{0}{1} \right) \otimes \left( c \cdot \vectordd{1}{0} + d \cdot \vectordd{0}{1} \right) = \ldots \\
        \ldots &= ac \vectordd{1}{0} \otimes \vectordd{1}{0} + ad \vectordd{1}{0} \otimes \vectordd{0}{1} + bc \vectordd{0}{1} \otimes \vectordd{1}{0} + bd \vectordd{0}{1} \otimes \vectordd{0}{1} \underset{\phi}{\cong} \ldots \\
        \ldots & \underset{\phi}{\cong} \begin{bmatrix} ac & ad \\ bc & bd \end{bmatrix}
    \end{split}
    \end{equation}

    Es decir:

    \begin{equation}
    \begin{split}
        \phi\left( \vectordd{a}{b} \otimes \vectordd{c}{d} \right) & = \begin{bmatrix} ac & ad \\ bc & bd \end{bmatrix} = \\
        & = \vectordd{a}{d} \begin{pmatrix} c & d \end{pmatrix}
    \end{split}
    \end{equation}

    O lo que es lo mismo:

    \begin{equation}
    \begin{split}
        \phi(v \otimes w) = v w^T
    \end{split}
    \end{equation}

    Esto coincide con la forma de definir los tensores que se hace en \cite{matematicas:principal}, y que introducimos en \customref{sec:otra_forma_tensores}.
\end{ejemplo}

Veamos ahora un ejemplo que nos clarifique la naturaleza del espacio $\R^N \otimes \mathbb{V}$:

\begin{ejemplo}
    Sea ahora $\Omega := \R^N \otimes \mathbb{V}$, con $N$ un natural cualquiera y $\mathbb{V}$ un espacio vectorial real. Consideramos su base usual:

    $$\mathbb{B}_{\R^N} := \left\{\vectorn{1}{0}{0}, \vectorn{0}{1}{0}, \ldots, \vectorn{0}{0}{1} \right\} = \left\{e_1, e_2, \ldots, e_N \right\}$$

    Ahora, veamos cómo podemos manipular un tensor cualquiera del espacio $\Omega$:

    \begin{equation}
    \begin{split}
        v \otimes w &= (\lambda_1 e_1 + \lambda_2 e_2 + \ldots \lambda_N e_N) \otimes w \eqtext{3.} \ldots \\
        \ldots &= \lambda_1 e_1 \otimes w + \lambda_2 e_2 \otimes w+ \ldots \lambda_N e_N \otimes w \eqtext{1., 2.} \ldots \\
        \ldots &= e_1 \otimes \lambda_1 w + e_2 \otimes \lambda_2 w + \ldots e_N \otimes \lambda_N w = \ldots \\
        \ldots &= e_1 \otimes w_1 + e_2 \otimes w_2 + \ldots e_N \otimes w_N \text{   con   } w_i \in \mathbb{V}
    \end{split}
    \end{equation}

    Por lo tanto, tenemos que $\forall v \in \R^N$, $\forall w \in \mathbb{V}$, $\exists w_1, \ldots w_N \in \mathbb{V}$ tal que:

    $$v \otimes w = e_1 \otimes w_1 + e_2 \otimes w_2 + \ldots + e_N \otimes w_N$$

    Luego, como hemos hecho en un ejemplo anterior, podemos definir el siguiente isomorfimso:

    \begin{equation}
    \begin{split}
        \phi: \R^N \otimes \mathbb{V} &\to \mathbb{V}^N \\
        v \otimes w = e_1 \otimes w_1 + e_2 \otimes w_2 + \ldots + e_N \otimes w_N &\mapsto \vectorn{w_1}{w_2}{w_N}
    \end{split}
    \end{equation}


\end{ejemplo}

\subsection{Otra forma de ver los tensores} \label{sec:otra_forma_tensores}

% TODO -- especificar que es lo que hemos visto anteriormente
Por lo visto anteriormente, tenemos una forma más concreta de entender los tensores.

Podemos ver un tensor $\mathcal{A} \in \R^{M_1, \ldots, M_N}$ como un \textit{array} multidimensional. La siguiente notación será muy útil a lo largo de este trabajo:

\begin{itemize}
    \item Modos: cada una de las entradas $d_1, \ldots, d_N$ que podemos usar para indexar los elementos del tensor
    \item Orden: el número de modos del tensor. En el caso de nuestro tensor $\mathcal{A}$, tenemos $N$ modos, y por tanto ese es su orden
    \item Dimensión: el número de valores que puede tomar cada uno de los modos. Por lo tanto, si en el primer modo $d_i$ puede tomar valores en $\deltaset{M}$, diremos que el modo $i$-ésimo tiene dimensión $M$.
        \begin{itemize}
            \item Un tensor puede tener distintas dimensiones en cada uno de los modos, o tener la misma dimensión para todos los modos
            \item Por tanto, sería más correcto hablar de \textit{"dimensiones de los modos"} que de \textit{"dimensión de un tensor"}, pero en ocasiones se abusa del lenguaje
        \end{itemize}
\end{itemize}

Ahora, respecto al \textbf{producto tensorial}, por los isomorfismos que hemos introducido previamente (TODO -- hay que desarrollar estos isomorfismos), podemos ver el producto tensorial entre dos tensores reales de una forma más sencilla.

Sean $\mathcal{A}, \mathcal{B}$ dos tensores de órdenes $P, Q$ respectivamente. Entonces el producto tensorial de estos dos, que ya sabemos que se denota como $\mathcal{A} \otimes \mathcal{B}$, es un tensor de orden $P + Q$ cuyos elementos se pueden expresar como:

$$(A \otimes B)d_1, \ldots d_{P + Q} = A_{d_1, \ldots, d_P} \cdot B_{d_{P + 1}, \ldots, d_{P + Q}}$$

En el caso de que tengamos dos vectores $\mathbf{u} \in \R^{N_1}, \mathbf{v} \in \R^{N_2}$, tenemos que $\mathbf{u} \otimes \mathbf{v} = \mathbf{u} \mathbf{v}^T$. Esto coincide con lo que vimos en \customref{ejemplo:R2xR2}, a partir de un isomorfismo natural.

\subsection{Descomposición CANDECOMP/PARAFAC}

Como ya se ha comentado en \customref{ch:introduccion}, las descomposiciones tensoriales van a ser fundamentales en este estudio. Así que empezamos con la descomposición mas sencilla de las dos. Comenzamos con la siguiente propiedad:

\begin{proposicion}[Producto tensorial de dos vectores]
    Sean $\mathbf{v} \in R^{N_1}, \mathbf{w} \in \R^{N_2}$ dos vectores. Entonces su producto tensorial puede expresarse como:

    $$\mathbf{v} \otimes \mathbf{w} = \mathbf{v} \mathbf{w}^T \in \matrices{N_1}{N_2}$$

    Además, esta matriz es de rango uno.
\end{proposicion}

% TODO -- probar esta proposicion, que la tengo en las notas a mano
\begin{proof}

TODO -- hay que probarlo!

\end{proof}

$\customqed$

Con esto, podemos definir el siguiente tipo de tensores:

% TODO -- esta definicion esta repetida de lo que ya teniamos
\begin{definicion}[Tensores puros]
    Un tensor $\mathcal{A}$ se dice que es \textbf{puro} (o también \textbf{elemental}) cuando es de la forma $\mathcal{A} = \otimes_{i = 1}^N \mathbf{v}^{(i)}$ con $\mathbf{v}^{(i)} \in \R^{N_i},\ \forall i \in \deltaset{N}$.
\end{definicion}

Esto nos da las herramientas necesarias para probar la descomposición buscada:

\begin{proposicion}[Descomposición \textit{CP}]
    Todo tensor $\mathcal{A}$ puede ser expresado como la suma de tensores elementales, es decir:

    \begin{equation} \label{eq:cp_decomp}
        \mathcal{A} = \sum_{i = 1}^Z \mathbf{v}_z^{(1)} \otimes \ldots \mathbf{v}_z^{(N)},\qquad
        \mathbf{v}_z^{(i)} \in \R^{M_i}, \forall z \in \deltaset{Z}, \forall i \in \deltaset{N}
    \end{equation}

\end{proposicion}

A la descomposición anterior se le llama \textbf{descomposición \textit{CANDECOMP/PARAFAC}}, o abreviadamente, \textbf{descomposición \textit{CP}}

\begin{proof}

TODO -- hay que demostrarlo

\end{proof}

$\customqed$

A partir de esto, hay que tener en cuenta las siguientes observaciones:

\begin{enumerate}
    % TODO -- colocar el valor real de $Z$
    \item En la demostración, hemos tomado $Z = TODO$ para asegurarnos de la existencia de una tal descomposición. Sin embargo, es razonable pensar que existirán combinaciones de vectores con las que podamos tomar un valor de $Z$ menor. Esto motivará la definición que haremos más adelante de \textit{rango CP}.
    \item Todos los sumandos tienen el mismo número de vectores con los que hacemos el producto tensorial. Además, este número es el orden del tensor que estamos descomponiendo
\end{enumerate}

Con esto, hemos motivado la siguiente definición:

\begin{definicion}[Rango \textit{CP}]
    Dado un tensor $\mathcal{A}$, se define su rango \textit{CP} como el mínimo valor de $Z$ para el cual la ecuación \eqref{eq:cp_decomp} se mantiene
\end{definicion}

Una propiedad interesante es la siguiente:

\begin{proposicion}[]
    Para un tensor de orden dos (que podemos ver como una matriz, por los isomorfismos previamente introducidos), su rango \textit{CP} coincide con el rango matricial usual
\end{proposicion}

\begin{proof}

TODO -- hay que demostrarlo

\end{proof}

\subsection{Propiedad Universal del producto tensorial}

El siguiente teorema será de gran utilidad a la hora de entender la naturaleza del producto tensorial entre dos espacios vectoriales.

\begin{teorema}[Propiedad Universal del producto tensorial] Sean $\mathbb{V}, \mathbb{W}$ dos espacios vectoriales. Su producto tensorial $\mathbb{V} \otimes \mathbb{W}$ es un espacio vectorial con una aplicación bilineal:

\begin{equation}
\begin{split}
    \otimes : \mathbb{V} \times \mathbb{W} &\to \mathbb{V} \otimes \mathbb{V} \\
    v, w & \mapsto v \otimes w
\end{split}
\end{equation}

de forma que:

\begin{equation}
    \forall h: \mathbb{V} \times \mathbb{W} \to \mathbb{Z} \text{  bilineal  } \exists! \hat{h}: \mathbb{V} \otimes \mathbb{W} \to \mathbb{Z} \text{  lineal, verificando que: }
\end{equation}

\begin{equation}
    h = \hat{h} \circ \otimes
\end{equation}

es decir:

\begin{equation}
    h(u, v) = \hat{h}(u \otimes v); \text{   } \forall \mathbf{u} \in \mathbb{V}, \forall \mathbf{v} \in \mathbb{W}
\end{equation}

Esto se resume en que el siguiente diagrama es conmutativo:

\begin{equation}
\begin{tikzcd}
    \mathbb{V} \times \mathbb{W} \ar{r}{h} \ar{d}[left]{\otimes} & \mathbb{Z} \\
    \mathbb{V} \otimes \mathbb{W} \ar[dashed]{ur}[right, below]{\hat{h}}
\end{tikzcd}
\end{equation}


\end{teorema}

Es decir, dada una aplicación bilineal en el producto cartesiano de dos espacios vectoriales, podemos asociar unívocamente una aplicación lineal en el producto tensorial de los dos espacios vectoriales. Esto sigue siendo cierto para aplicaciones multilineales en el producto cartesiano de un número arbitrario de espacios vectoriales.

Este teorema, que se puede probar a partir de todo lo que hemos visto hasta ahora, sirve para dar una definición no constructiva del producto tensorial de dos espacios vectoriales. A diferencia de lo que pasaba con la definición alternativa que se puede dar usando \customref{th:base_prod_tensorial}, esta definición no depende de ninguna base, y por lo tanto es igual de general que la definición por la que hemos optado.

\newpage
\section{Teoría de la medida}

\endinput
