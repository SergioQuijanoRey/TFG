\chapter{Herramientas matemáticas fundamentales}

\section{Tensores}

\subsection{Notación}

Seguiremos en parte la notación del trabajo principal \cite{matematicas:principal}, aunque introducimos pequeños cambios.

Denotaremos a los vectores con tipografía en negrita, tal que $\mathbf{v} \in \R^N$. Las coordenadas de dicho vector se denotarán como $v_i$ con $i \in \deltaset{n}$, donde $\deltaset{n} := \{1, \ldots, n\}$.

Para denotar a los tensores (que introduciremos más adelante), usaremos la tipografía caligráfica. Por ejemplo $\mathcal{A} \in \R^{M_1 \times \ldots M_N}$. Cada una de las entradas de dicho tensor (más adelante, en \customref{sec:deftensor}, veremos que significa esto) serán denotadas como $\mathcal{A}_{d_1, \ldots, d_N} \in \R$.

Al espacio de matrices de dimensiones $p, q$ lo denotaremos, como es usual, como $\matrices{p}{q}$.

Al producto tensorial entre dos tensores $\mathcal{A}, \mathcal{B}$ lo denotaremos como $\mathcal{A} \otimes \mathcal{B}$. Dado un conjunto de vectores $\mathbf{v}_1, \ldots, \mathbf{v}_N$, denotaremos su producto tensorial $\mathbf{v}_1 \otimes \ldots \otimes {v}_N$ como $\otimes_{i = 1}^N \mathbf{v}_i$.

Al producto de Kronecker entre dos matrices $A, B$ lo denotaremos como $A \odot B$.

\subsection{Definición del producto tensorial} \label{sec:deftensor}

Dados dos espacios vectoriales reales (aunque podría realizarse la construcción sobre otro cuerpo) $\mathbb{V}, \mathbb{W}$, queremos construir el espacio producto tensorial de estos espacios vectoriales, denotado como $\mathbb{V} \otimes \mathbb{W}$.

Para la construcción del producto tensorial de espacios vectoriales, necesitaremos primero introducir conceptos previos que serán fundamentales en la construcción del objeto matemático que buscamos.

\subsubsection{Producto formal de dos espacios vectoriales}

% TODO -- la notacion span es americana, buscar la española en los apuntes de
% geometria I
\begin{definicion}[Producto formal de dos espacios vectoriales]
    Sean $\mathbb{V}, \mathbb{W}$ dos espacios vectoriales reales. Se define su \textbf{producto formal} como:

    \begin{equation}
        \mathbb{V} \ast \mathbb{W} := span_{\mathbb{R}} \{v \ast w / v \in \mathbb{V}, w \in \mathbb{W} \}
    \end{equation}

    Donde $span$ denota el conjunto formado por todas las combinaciones lineales finitas de los elementos del conjunto, es decir,

    \begin{equation}
        span_{\mathbb{R}}(A) := \{ \sum_{k = 1}^n \alpha_i a_i / n \in \N, \alpha_i \in \R, a_i \in A \}
    \end{equation}

    Hay que tener en cuenta que en $v \ast w$, $\ast$ es un símbolo con el que no sabemos operar. Por tanto, ahora mismo no sabemos simplificar muchas expresiones en este espacio.
\end{definicion}

Es claro que por ser $\mathbb{V}, \mathbb{W}$ espacios vectoriales, y estar tomando combinaciones lineales finitas, $\mathbb{V} \ast \mathbb{W}$ es un espacio vectorial.

Veamos ahora algunos ejemplos para familiarizarnos con este espacio.

\underline{\textbf{Ejemplo}:} Sea $\Omega := R^2 \ast R^3$. Entonces tenemos que:

\begin{equation}
    \Omega = span \{
        \begin{pmatrix}
            a \\
            b
        \end{pmatrix}
        \ast
        \begin{pmatrix}
            x \\
            y \\
            z
        \end{pmatrix}
        / a, b, x, y, z \in \R
    \}
\end{equation}



\subsubsection{Espacio cociente del producto formal}

\subsubsection{Espacio cociente del producto formal}

\subsection{Otra forma de ver los tensores}

% TODO -- especificar que es lo que hemos visto anteriormente
Por lo visto anteriormente, tenemos una forma más concreta de entender los tensores.

Podemos ver un tensor $\mathcal{A} \in \R^{M_1, \ldots, M_N}$ como un \textit{array} multidimensional. La siguiente notación será muy útil a lo largo de este trabajo:

\begin{itemize}
    \item Modos: cada una de las entradas $d_1, \ldots, d_N$ que podemos usar para indexar los elementos del tensor
    \item Orden: el número de modos del tensor. En el caso de nuestro tensor $\mathcal{A}$, tenemos $N$ modos, y por tanto ese es su orden
    \item Dimensión: el número de valores que puede tomar cada uno de los modos. Por lo tanto, si en el primer modo $d_i$ puede tomar valores en $\deltaset{M}$, diremos que el modo $i$-ésimo tiene dimensión $M$.
        \begin{itemize}
            \item Un tensor puede tener distintas dimensiones en cada uno de los modos, o tener la misma dimensión para todos los modos
            \item Por tanto, sería más correcto hablar de \textit{"dimensiones de los modos"} que de \textit{"dimensión de un tensor"}, pero en ocasiones se abusa del lenguaje
        \end{itemize}
\end{itemize}

Ahora, respecto al \textbf{producto tensorial}, por los isomorfismos que hemos introducido previamente (TODO -- hay que desarrollar estos isomorfismos), podemos ver el producto tensorial entre dos tensores reales de una forma más sencilla.

Sean $\mathcal{A}, \mathcal{B}$ dos tensores de órdenes $P, Q$ respectivamente. Entonces el producto tensorial de estos dos, que ya sabemos que se denota como $\mathcal{A} \otimes \mathcal{B}$, es un tensor de orden $P + Q$ cuyos elementos se pueden expresar como:

$$(A \otimes B)d_1, \ldots d_{P + Q} = A_{d_1, \ldots, d_P} \cdot B_{d_{P + 1}, \ldots, d_{P + Q}}$$

\subsection{Descomposición CANDECOMP/PARAFAC}

Como ya se ha comentado en \customref{ch:introduccion}, las descomposiciones tensoriales van a ser fundamentales en este estudio. Así que empezamos con la descomposición mas sencilla de las dos. Comenzamos con la siguiente propiedad:

\begin{proposicion}[Producto tensorial de dos vectores]
    Sean $\mathbf{v} \in R^{N_1}, \mathbf{w} \in \R^{N_2}$ dos vectores. Entonces su producto tensorial puede expresarse como:

    $$\mathbf{v} \otimes \mathbf{w} = \mathbf{v} \mathbf{w}^T \in \matrices{N_1}{N_2}$$

    Además, esta matriz es de rango uno.
\end{proposicion}

% TODO -- probar esta proposicion, que la tengo en las notas a mano
\demostracion

TODO -- hay que probarlo!

$\customqed$

Con esto, podemos definir el siguiente tipo de tensores:

\begin{definicion}[Tensores elementales]
    Un tensor $\mathcal{A}$ se dice que es \textbf{elemental} (o también se le puede llamar puro) cuando es de la forma $\mathcal{A} = \otimes_{i = 1}^N \mathbf{v}^{(i)}$ con $\mathbf{v}^{(i)} \in \R^{N_i},\ \forall i \in \deltaset{N}$.
\end{definicion}

Esto nos da las herramientas necesarias para probar la descomposición buscada:

\begin{proposicion}[Descomposición \textit{CP}]
    Todo tensor $\mathcal{A}$ puede ser expresado como la suma de tensores elementales, es decir:

    \begin{equation} \label{eq:cp_decomp}
        \mathcal{A} = \sum_{i = 1}^Z \mathbf{v}_z^{(1)} \otimes \ldots \mathbf{v}_z^{(N)},\qquad
        \mathbf{v}_z^{(i)} \in \R^{M_i}, \forall z \in \deltaset{Z}, \forall i \in \deltaset{N}
    \end{equation}

\end{proposicion}

A la descomposición anterior se le llama \textbf{descomposición \textit{CANDECOMP/PARAFAC}}, o abreviadamente, \textbf{descomposición \textit{CP}}

\demostracion

TODO -- hay que demostrarlo

$\customqed$

A partir de esto, hay que tener en cuenta las siguientes observaciones:

\begin{enumerate}
    % TODO -- colocar el valor real de $Z$
    \item En la demostración, hemos tomado $Z = TODO$ para asegurarnos de la existencia de una tal descomposición. Sin embargo, es razonable pensar que existirán combinaciones de vectores con las que podamos tomar un valor de $Z$ menor. Esto motivará la definición que haremos más adelante de \textit{rango CP}.
    \item Todos los sumandos tienen el mismo número de vectores con los que hacemos el producto tensorial. Además, este número es el orden del tensor que estamos descomponiendo
\end{enumerate}

Con esto, hemos motivado la siguiente definición:

\begin{definicion}[Rango \textit{CP}]
    Dado un tensor $\mathcal{A}$, se define su rango \textit{CP} como el mínimo valor de $Z$ para el cual la ecuación \eqref{eq:cp_decomp} se mantiene
\end{definicion}

Una propiedad interesante es la siguiente:

\begin{proposicion}[]
    Para un tensor de orden dos (que podemos ver como una matriz, por los isomorfismos previamente introducidos), su rango \textit{CP} coincide con el rango matricial usual
\end{proposicion}

\demostracion

TODO -- hay que demostrarlo

$\customqed$

% TODO -- aqui igual deberia hablar de tensores simetricos y descomposiciones simetricas CP

\newpage
\section{Teoría de la medida}

\endinput

