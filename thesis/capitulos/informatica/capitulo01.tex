\chapter{Introducción}

% TODO -- estructura que le quiero dar al trabajo
% 1. Introduccion
% 1. Hablar del problema de retrieval de imagenes, y por qué este problema es relevante
% 1. Hablar del dataset que hemos empleado y de otros datasets que hemos considerado
% 1. Explicar el triplet loss, ventajas que se plantean en el paper de referencia
% 1. Desarrollo de Software
%     1. Explicar las herramientas usadas (entornos de python, librerias principales)
%     2. Explicar github CI CD
%     3. Explicar como accedemos al servidor
%     4. Explicar toda la estructura del codigo, patrones de diseños, sacarme partido aqui que es donde mas fuerte estoy
% 1. Hablar de la aproximacion al problema, iterativa, usando MNIST, LFW, luego CACD + FG-Net
%     - Esto se puede meter en algo asi como planificacion. Lo justifico diciendo que habia mucho codigo nuevo que implementar, y que para iterar de forma rapida iba resolviendo problemas cada vez mas complejos y mas pesados en lo que tamaño de datset se refiere
% 1. Analisis de los malos resultados
% 1. Proponer mejoras a estos problemas

% TODO -- tomar de referencia la guia de Pablo Mesejo: https://drive.google.com/drive/folders/1BGI7zUp0kiZ6ufCH1UlhzvcFPdZ5NFWN
% A partir de esta, importante:

% 1. Dar prioridad al resumen, introduccion (descripcion del problema, motivacion, objetivos) y las conclusiones (**"deben estar perfectas"**)
% 2. Introducir una figura de SCOPUS
